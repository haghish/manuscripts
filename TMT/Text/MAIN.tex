\documentclass[]{article}

% APA citation style 
\usepackage{apacite}
\bibliographystyle{apacite} 

%opening
\title{: \\ }
\title{%
	Trail Making Test (TMT) App: \\
	\large An R package and Shiny application for normative interpretation of TMT test}

\author{E. F. Haghish}

\begin{document}

\maketitle

\begin{abstract}
	The Trail Making Test (TMT) is one of the common neuropsychological tests due to its simplicity and its sensitivity to brain damage. The test measures the performance time for two visual-conceptual and visual-motor tracking tasks. Knowing that different factors such as age, level of education, and general intelligence influence the results, different norms and cut-off values have been published for making sense of the results of the TMT test. The current manuscript reviews some of the frequently used norms and cut-off values and introduces a software and an R package for interpreting the TMT performance time based on these norms.  
\end{abstract}

\section{Introduction}
	The Trail Making Test (TMT, 1944) was introduces within the Army Individual Test Battery in 1944. Due to the simplicity of administration and sensitivity to brain damage, it has become one of the most popular neuropsychology tests. 
	
	
	•	What is the test looking like?
	•	How it relates to brain damage and what information it provides?
	•	How the results of the test are interpreted?
	•	What are we doing in this manuscript
	o	What is wrong with the previous norms?
	
\end{document}
